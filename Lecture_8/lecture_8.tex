\documentclass[11.5pt, paper=a4]{article}

\usepackage[utf8]{inputenc}
\usepackage[english]{babel}
\usepackage[T1]{fontenc}

\usepackage{amsmath, amssymb, amscd, amsthm, amsfonts, mathtools}
\usepackage[left=2cm, right=2cm, top=1.5cm]{geometry}

\usepackage{graphicx}
\usepackage{hyperref}
\usepackage{physics}
\usepackage{tikz}
\usepackage{url}
\usepackage[square,numbers]{natbib}
\usepackage{tabularx}

\usepackage{braket}
\usepackage{thmtools}
\usepackage{float}

%%% Theorem Style
\theoremstyle{definition}
\newtheorem{theorem}{Theorem}[section]
\newtheorem{definition}[theorem]{Definition}
\newtheorem{lemma}[theorem]{Lemma}
\newtheorem{conjecture}[theorem]{Conjecture}
\newtheorem{corollary}[theorem]{Corollary}

\numberwithin{theorem}{section}

%% Autoref prefixes
\renewcommand{\sectionautorefname}{Section}
\renewcommand{\subsectionautorefname}{Section}
\renewcommand{\subsubsectionautorefname}{Section}
\renewcommand{\figureautorefname}{Figure}
\def\theoremautorefname{Theorem}
\def\lemmaautorefname{Lemma}
\def\definitionautorefname{Definition}
\def\conjectureautorefname{Conjecture}
\def\algorithmautorefname{Algorithm}

%% Writing algorithms

\usepackage{algorithm} % captioning 
\usepackage{algpseudocode}

% \def\NoNumber#1{{\def\alglinenumber##1{}\State #1}\addtocounter{ALG@line}{-1}}

\graphicspath{{./Lecture_8/images/}}

\title{Quantum Algorithms, Spring 2022: Lecture 8 Scribe}

\author{Pahulpreet Singh \and Shreyas Pradhan}

\date{February 04, 2022}

\begin{document}

\maketitle

\section{Recap}

% TODO: Simon's algorithm – Quantum Implementation

\section{Q.F.T. and its applications}

\subsection{Elementary Concepts}

\subsubsection{Complex Roots of Unity}

\begin{align*}
    \omega^N = 1  \hspace{100px}                  & \omega = e^{\displaystyle \frac{-2\pi\iota}{N}} \\
    1 + \omega + \omega^2 + \cdots + \omega^{N-1} & = 0
\end{align*}

% Insert Example Figure

\subsubsection{Discrete Fourier Transform}

A complex vector $\big(x_0, x_1, \cdots, x_{N-1} \big)$ is transformed to another complex vector
$\big( y_0, y_1, \cdots, y_{N-1} \big)$ such that:

\begin{equation}
    y_k = \sum_{j=0}^{N-1} x_j \cdot \omega^{jk}
\end{equation}

\subsection{Quantum Fourier Transform}

For Q.F.T. (Quantum Fourier Transform), the same transformation is carried out, except that it maps
quantum states to quantum states.

\paragraph{Example}
Say $N = 2^n$ and $\{ \ket{0}, \ket{1}, \cdots, \ket{N-1} \}$ are computational basis states. Then
QFT on a state $\ket{j}$ is defined as:

\[
    \ket{j} \xmapsto{QFT} \displaystyle \frac{1}{\sqrt{N}} \sum_{k=0}^{N-1} \omega^{jk} \ket{k}
\]

where $\displaystyle \frac{1}{\sqrt{N}}$ is the normalization factor. Generalizing this to an ensemble of states,

\begin{align}
    \sum_{j=0}^{N-1} x_j\ket{j} \xmapsto{QFT} & \sum_{k=0}^{N-1} y_k \ket{k}                   \\
                                              & \big[ \text{where } y_k = \sum_{j=0}^{N-1} x_j
        \frac{\omega^{jk}}{\sqrt{N}} \big] \nonumber
\end{align}


We define $F_N$ as the unitary operator for QFT modulo $N$.

\[
    F_N\ket{j} \xmapsto{} \displaystyle \frac{1}{\sqrt{N}} \sum_{k=0}^{N-1} \omega^{jk} \ket{k}
\]

And the $(j, k)^{th}$ entry of the matrix is given by:

\[
    (F_N)_{j,k} = \braket{k|F_N|j} = \displaystyle \frac{\omega^{jk}}{\sqrt{N}}
\]

From this, we can compute $F_N$ as:

\[
    F_N = \displaystyle \frac{1}{\sqrt{N}} \begin{pmatrix}
        1      & 1            & 1               & \cdots & 1                \\
        1      & \omega       & \omega^2        & \cdots & \omega^{N-1}     \\
        1      & \omega^2     & \omega^4        & \cdots & \omega^{2(N-1)}  \\
        \vdots &              &                 & \ddots & \vdots           \\
        1      & \omega^{N-1} & \omega^{2(N-1)} & \cdots & \omega^{(N-1)^2} \\
    \end{pmatrix}
\]

\begin{enumerate}
    \item Classically, for D.F.T., we need $O(N \log{N})$ steps / operations to F.F.T. (Fast Fourier Transform).
    \item For QFT, we shall require $O(\log^2{N})$ elementary gates.
    \item Instead of a vector, QFT outputs a quantum state!
    \item Classically, we have access to all the $y_k$ in the vector. This is not the case for QFT.
    \item At the end of QFT, we need to make a measurement. We observe some $\ket{k}$ with the probability $|y_k|^2$.
    \item When we sample from the output state, we do so according to the Fourier Transform coefficients. (This is called Fourier Sampling)
\end{enumerate}




\bibliographystyle{plainnat}
\nocite{*}
\bibliography{references}

\end{document}

