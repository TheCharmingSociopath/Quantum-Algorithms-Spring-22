\documentclass[11.5pt, paper=a4]{article}

\usepackage[utf8]{inputenc}
\usepackage[english]{babel}
\usepackage[T1]{fontenc}

\usepackage{tikz}
\usetikzlibrary{quantikz}
\usepackage{amsmath, amssymb, amscd, amsthm, amsfonts, mathtools}
\usepackage[left=2cm, right=2cm, top=1.5cm]{geometry}

\usepackage{graphicx}
\usepackage{hyperref}
\usepackage{physics}
\usepackage{tikz}
\usepackage{url}
\usepackage[square,numbers]{natbib}
\usepackage{tabularx}

\usepackage{braket}
\usepackage{thmtools}
\usepackage{float}

%%% Theorem Style
\theoremstyle{definition}
\newtheorem{theorem}{Theorem}[section]
\newtheorem{definition}[theorem]{Definition}
\newtheorem{lemma}[theorem]{Lemma}
\newtheorem{conjecture}[theorem]{Conjecture}
\newtheorem{corollary}[theorem]{Corollary}

\numberwithin{theorem}{section}

%% Autoref prefixes
\renewcommand{\sectionautorefname}{Section}
\renewcommand{\subsectionautorefname}{Section}
\renewcommand{\subsubsectionautorefname}{Section}
\renewcommand{\figureautorefname}{Figure}
\def\theoremautorefname{Theorem}
\def\lemmaautorefname{Lemma}
\def\definitionautorefname{Definition}
\def\conjectureautorefname{Conjecture}
\def\algorithmautorefname{Algorithm}

%% Writing algorithms
\setlength{\parindent}{0mm}
\usepackage{algorithm} % captioning 
\usepackage{algpseudocode}

% \def\NoNumber#1{{\def\alglinenumber##1{}\State #1}\addtocounter{ALG@line}{-1}}

\graphicspath{{./Lecture_X/images/}}

\title{Quantum Algorithms, Spring 2022: Lecture 19 Scribe}

\author{Shreyas Pradhan, Pahulpreet Singh}

\date{March 25, 2022}

\begin{document}

\maketitle

\section{Recap}

\subsection{Grover's Quantum Search Algorithm}

\subsubsection*{Aim}
Given a black box $U_f$ for the Boolean function $f \rightarrow \{ 0,1 \} $, where $X = \{ x_1, x_2, ..., x_N \}$, how many queries to $U_f$ are needed to find $x^* \in X$, such that $f(x^*) = 1$.

\vspace{5mm}
$M (<< N)$ is the number of elements in $X$ such that $f(x) = 1$.

\subsubsection*{Quantum Algorithm}

\begin{center}
\begin{quantikz}
\lstick{$\ket 0 ^ {\otimes n}$} & \gate[wires=1]{H^{\otimes n}} & \gate{U_f^{\pm}} & \gate{D} & \qw\ \ldots\ & \gate{U_f^{\pm}} & \gate{D} & \meter{} & \cw
\end{quantikz}
\end{center}

We define the states that are marked as 
\begin{equation}
|w \rangle = \dfrac{1}{\sqrt{M}} \sum_{\{x: f(x) = 1 \}} |x \rangle
\end{equation}
and the states that are not marked as
\begin{equation}
|s_{\bar{w}} \rangle = \dfrac{1}{\sqrt{N - M}} \sum_{\{x: f(x) = 0 \}} |x \rangle
\end{equation}
Hence we can write out initial state as
\begin{equation}
|s \rangle = \cos{(\dfrac{\theta}{2})} |s_{\bar{w}} \rangle + \sin{(\dfrac{\theta}{2})} |w \rangle
\end{equation}
where $\sin{(\dfrac{\theta}{2})} = \sqrt{\dfrac{M}{N}}$.

\vspace{5mm}

Now we define $G = D U_f^{\pm}$, where
\begin{equation}
U_f : I - 2|w\rangle \langle w| = - R_w
\end{equation}
is the oracle that implements the marking function (reflection about |$s_{\bar{w}}\rangle$)
\begin{equation}
D : 2|s\rangle \langle s| - I = R_s
\end{equation}
and D is the diffuser (reflection about |$s \rangle$). \\
Hence $G = - R_s R_w$ is the operation we need to apply multiple times.

\vspace{5mm}

In the 2-D subspace spanned by $\{ |s_{\bar{w}} \rangle, |w \rangle \}$,
\begin{equation}
G = \begin{pmatrix}
\cos{\theta} & -\sin{\theta} \\
\sin{\theta} & \cos{\theta}
\end{pmatrix}
\end{equation}
(as the product of two reflections $\equiv$ Rotation by $\theta$)

\vspace{5mm}

We proved that for $K = \dfrac{\pi}{4}\sqrt{\dfrac{N}{M}} - \dfrac{1}{2}$, 
\begin{equation}
G^K |s \rangle = | w \rangle
\end{equation}

\section{What if we don't know $M$?}
\subsection{Guessing $M$}
\begin{itemize}
\item Run Grover's algorithm for different guesses of $M (= 1, 2, 4, ..., 2^{\log_2N})$

\item If $M'$ is the actual number of solutions, $\exists$ a guess $M: \dfrac{M'}{2} \leq M \leq 2M'$

\item For this guess, the success probability is a constant

\item Expected number of queries to $U_f^{\pm}: O(\sqrt{N})$
\end{itemize}

\subsection{Quantum Counting}
In the case that we do not know what $M$ is, we can estimate $M$ using Quantum Phase Estimation applied on $G$.

\vspace{5mm}

In the 2D invariant space,

\begin{equation}
    G = \begin{pmatrix}
    \cos{\theta} & -\sin{\theta} \\
    \sin{\theta} & \cos{\theta}
    \end{pmatrix}
\end{equation}
where $\dfrac{\theta}{2} = \sin^{-1}(\sqrt{\dfrac{M}{N}})$


Eigenvalues and eigenvectors of $G$ are $\lambda_{\pm} = e^{\pm i \theta}$, $\ket{\lambda_{\pm}} = \dfrac{\ket{w} \pm i \ket{s_{\bar{w}}}}{\sqrt{2}}$

\subsubsection{Quantum Phase Estimation}

\textbf{Aim} \\
Given $U$ with eigenvalue $e^{i2 \pi \phi}$ and eigenstate $\ket{u}$ obtain $\tilde{\phi}$ such that $|\tilde{\phi} - \phi| \leq \delta$ with $p_{succ} \geq 1 - \epsilon$


\textbf{Time Complexity}: $O(\dfrac{T_u}{\delta \epsilon})$

\vspace{5mm}

Eigenvalues of G are
\begin{equation*}
    \lambda_{+} = e^{i \theta} = e^{i 2\pi (\frac{\theta}{2 \pi})}
\end{equation*}
\begin{equation*}
    \lambda_{-} = e^{-i \theta} = e^{i 2\pi (1 - \frac{\theta}{2 \pi})}
\end{equation*}
In our case, we need to find $\lambda_{\pm}$ to find $\theta$. However there is no way beforehand to prepare $\ket{\lambda_{\pm}}$.

\vspace{5mm}

\textbf{Idea} \\
To prepare state $\ket{s} = \cos(\frac{\theta}{2})\ket{s_{\bar{w}}} + \sin(\frac{\theta}{2})\ket{w}$

\vspace{5mm}

We can rewrite $\ket{s}$ in terms of $\ket{\lambda_{\pm}}$. 
\begin{equation*}
    |\langle \lambda_{\pm} | s \rangle|^2 = \frac{1}{2}
\end{equation*}
So, 
\begin{equation*}
 \ket{s} = \dfrac{e^{i \theta}}{\sqrt{2}} \ket{\lambda_+} - \dfrac{e^{-i \theta}}{\sqrt{2}} \ket{\lambda_-}   
\end{equation*}

We will now apply Quantum Phase Estimation with $G$ and $\ket{s}$
\begin{equation}
    \ket{0}^{\otimes t}\ket{s} \rightarrow  \dfrac{e^{i \theta}}{\sqrt{2}} \ket{\tilde{\theta_{+}}}\ket{\lambda_+} - \dfrac{e^{-i \theta}}{\sqrt{2}} \ket{\tilde{\theta_{-}}}\ket{\lambda_-}
\end{equation}

\textbf{Quantum Circuit for QPE}
\begin{center}
\begin{quantikz}
\lstick{$\ket 0 ^ {\otimes t}$} &[2mm] \gate[wires=1]{H^{\otimes t}} \qwbundle
{t} & \ctrl{1} & \gate{\text{QFT}^\dagger} & \meter{} \\ 
\lstick{$\ket 0 ^ {\otimes n}$} &[2mm] \gate[wires=1]{H^{\otimes n}} \qwbundle
{n} & \gate{G^{2^{t-1}}} & \qw & \qw
\end{quantikz}
\end{center}

We obtain some $\tilde{\theta}$ such that $|\tilde{\theta} - \theta| \leq \delta$ with probability $\geq 1 - \epsilon$. Here the total qubits used was $t = d + \lceil \log(\frac{1}{\epsilon}) \rceil$ and $\delta = 2^{-d}$.

\vspace{5mm}
Since, our state $\ket{s}$ was a superposition of $\ket{\lambda_+}$ and $\ket{\lambda_+}$, we can obtain either $\theta$ or $-\theta$ after QPE.
However, it does not matter which we obtain as for a good estimate of $M (<< N)$ we need
\begin{equation}
    \sin^2(\frac{\theta}{2}) = \frac{M}{N} \implies M = N \sin^2(\frac{\theta}{2})
\end{equation}
From QPE, we would estimate: $\tilde{M} = N \sin^2(\frac{\tilde{\theta}}{2})$

\subsubsection{Precision of estimated $\tilde{M}$}
\begin{equation}
    |\tilde{M} - M| = N |\sin^2(\frac{\tilde{\theta}}{2}) - \sin^2(\frac{\theta}{2})| \leq N |\sin(\frac{\tilde{\theta}}{2}) + \sin(\frac{\theta}{2})||\sin(\frac{\tilde{\theta}}{2}) - \sin(\frac{\theta}{2})|
\end{equation}

\textbf{Mean Value Theorem}
\begin{equation}
    \dfrac{|f(\theta_1) - f(\theta_2)|}{|\theta_1 - \theta_2|} = |f'(e)|
\end{equation}
for some $e \in [\theta_1, \theta_2]$

\vspace{5mm}
Hence, by MVT
\begin{equation}
    |\sin(\frac{\tilde{\theta}}{2}) - \sin(\frac{\theta}{2})| \leq \frac{1}{2}|\tilde{\theta} - \theta|
\end{equation}

Now,
\begin{align*}
    |\sin(\frac{\tilde{\theta}}{2}) + \sin(\frac{\theta}{2})| &\leq |\sin(\frac{\tilde{\theta}}{2})| + |\sin(\frac{\theta}{2})| \\
    &\leq |\sin(\frac{\theta + \delta}{2})| + |\sin(\frac{\theta}{2})| \\
    &\leq |\sin(\frac{\theta}{2})\cos(\frac{\delta}{2}) + \cos(\frac{\theta}{2})\sin(\frac{\delta}{2})| + |\sin(\frac{\theta}{2})|\\
    &\leq |\sin(\frac{\theta}{2})\cos(\frac{\delta}{2})| + |\cos(\frac{\theta}{2})\sin(\frac{\delta}{2})| + |\sin(\frac{\theta}{2})|\\
    &\leq 2 \sin(\frac{\theta}{2}) + |\frac{\delta}{2}|
\end{align*}

\begin{align*}
    \therefore |\tilde{M} - M| &\leq (2 \sin(\frac{\theta}{2}) + |\frac{\delta}{2}|) |\tilde{\theta} - \theta| \frac{N}{2} \\
    &\leq N (2 \sin(\frac{\theta}{2}) + |\frac{\delta}{2}|) \frac{|\delta|}{2} \\
    &\leq N (2 \frac{M}{N} + |\frac{\delta}{2}|) \frac{|\delta|}{2} \\
    &= \sqrt{MN} |\delta| + \frac{N|\delta|^2}{4}
\end{align*}

If we choose $|\delta| = \frac{1}{\sqrt{N}}$, then
\begin{equation}
    |\tilde{M} - M| \leq \sqrt{M} + \frac{1}{4}
\end{equation}

Hence, we can estimate $M$ to a constant multiplicative precision
\begin{equation}
    \tilde{M} = M (1 \pm O(\frac{1}{\sqrt{M}}))
\end{equation}
which is good enough to obtain a constant success probability for Grover's Algorithm. 

Precision of QPE: $\delta = \frac{1}{\sqrt{N}}$
\begin{equation*}
\delta = 2^{-d} \implies d = \frac{n}{2}   \ \text{qubits}
\end{equation*}


Hence, if we demand $p_{succ} \geq 1 - \epsilon \implies t = \frac{n}{2} + \lceil \log(\frac{1}{\epsilon}) \rceil$

\textbf{Query Complexity}: $O(\sqrt{N})$ which is advantageous.

\vspace{5mm}
Quantum counting also allows us to determine if $M=0$ (when $|\tilde{M}| < \frac{1}{4}$).

\section{Optimality of Quantum Search}

Grover's algorithm provides an optimal separation in query complexity for the quantum search problem.

\vspace{5mm}
Consider that there is one solution $w$.

The setting should be such that we are able to distinguish between any of the $N$ possible values of $w$  after measurement.

\vspace{5mm}
Let us consider a quantum circuit with $T$ queries:

\begin{equation}
    \ket{\psi_{T}^{w}} = U_{T}U_{w}U_{T-1}U_{w} ... U_{1}U_{w} \ket{s} = U{w, T} \ket{s}
\end{equation}
where $U_{w} = 1 - 2\ket{w}\bra{w}$

\vspace{5mm}
Now consider another quantum circuit that does not use $U_w$ but only uses $U_j$'s $T$ times. Applying this circuit to $\ket{s}$ we have:

\begin{equation}
    \ket{\psi_{T}} = U_{T}U_{T-1} ... U_{1} \ket{s}
\end{equation}

$U(w, T)$ interleaves $U_w$ and $U_j$'s just like Grover's algorithm. The `oracle oblivious' circuit is the extreme case, where we completely ignore the oracle.

\subsection*{Strategy}

We will bound the distance between $\ket{\psi_{T}^{w}}$ and $\ket{\psi_T}$. Any $\ket{w}$ can be a solution.

\vspace{5mm}
After the $T^{th}$ iteration, the distance between them:
\begin{equation}
    || \ket{\psi_{T}^{w}} - \ket{\psi_{T}} ||^2
\end{equation}

Since we want to consider all possible $\ket{w}$'s, we want to bound:
\begin{equation}
    D_T = \sum_{w \in \{0,1\}^n} || \ket{\psi_{T}^{w}} - \ket{\psi_{T}} ||^2
\end{equation}
    
We will prove an upper bound and a lower bound on $D_T$.

\subsection*{Claims}

\begin{enumerate}
    \item After $T$ iterations, $D_T \leq 4T^2$
    \item If the algorithm has to succeed with $p \geq \frac{1}{2}$ ,ie, $|\langle w \ket{\psi_{T}^{w}}| \geq \frac{1}{2}$ for any $w \in \{0,1\}^n$ that is marked, then $D_T = \Omega(N)$
\end{enumerate}

Claims 1 and 2 together imply that $T \geq c \sqrt{N}$ for the algorithm to succeed with a high probability.

% \bibliographystyle{plainnat}
% \bibliography{references}

\end{document}
